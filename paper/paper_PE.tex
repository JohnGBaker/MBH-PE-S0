\documentclass[aps,showpacs,twocolumn,prd,superscriptaddress,nofootinbib]{revtex4}

\usepackage{amsmath}
\usepackage{amsfonts}
\usepackage{amssymb}
\usepackage{latexsym}
\usepackage{graphicx}
\usepackage{bm}
%\usepackage{color}
\usepackage{enumerate}
\usepackage{ulem}

\usepackage{color}
\usepackage[usenames,dvipsnames,svgnames,table]{xcolor}
%\usepackage[colorlinks=true,
%            linkcolor=green,
%            urlcolor=blue,
%            citecolor=red]{hyperref}
\usepackage[colorlinks=true,
            linkcolor=YellowOrange,
            urlcolor=RoyalBlue,
            citecolor=RedViolet]{hyperref}

\newcommand{\be}{\begin{equation}}
\newcommand{\ee}{\end{equation}}
\newcommand{\bsub}{\begin{subequations}}
\newcommand{\esub}{\end{subequations}}
\newcommand\ud{{\mathrm{d}}}
\newcommand\uD{{\mathrm{D}}}
\newcommand\calO{{\mathcal{O}}}
\newcommand\bfx{\mathbf{x}}
\newcommand{\ov}[1]{\overline{#1}}
\newcommand{\ph}[1]{\phantom{#1}}
\newcommand{\cte}{\mathrm{cte}}
\newcommand{\nn}{\nonumber}
\newcommand{\hatk}{\hat{k}}
\newcommand{\Hz}{\,\mathrm{Hz}}
\newcommand{\sinc}{\,\mathrm{sinc}}
\newcommand{\Msol}{M_{\odot}}
\newcommand\calL{{\mathcal{L}}}
\newcommand\calA{{\mathcal{A}}}
\newcommand\betaL{{\beta_{L}}}
\newcommand\lambdaL{{\lambda_{L}}}
\newcommand\varphiL{{\varphi_{L}}}
\newcommand\psiL{{\psi_{L}}}

\newcommand{\SM}[1]{{\color{Blue} #1}}
\newcommand{\jgb}[1]{{\color{DarkGreen} #1}}

\begin{document}

\title{Bayesian methods for black hole merger parameter estimation with LISA.}

\author{John G. Baker}
\affiliation{Gravitational Astrophysics Laboratory, NASA Goddard Space Flight Center, 8800 Greenbelt Rd., Greenbelt, MD 20771, USA}
\author{Sylvain Marsat}
\affiliation{Department of Physics, University of Maryland, College Park, MD 20742, USA}
\affiliation{Gravitational Astrophysics Laboratory, NASA Goddard Space Flight Center, 8800 Greenbelt Rd., Greenbelt, MD 20771, USA}
\affiliation{Max Planck Institute for Gravitational Physics (Albert Einstein Institute), Am M\"uhlenberg 1, Potsdam-Golm, 14476, Germany}


\date{\today}

\begin{abstract}

[Abstract]

\end{abstract}

\pacs{
04.70.Bw, % classical black holes
04.80.Nn, % Gravitational wave detectors and experiments
95.30.Sf, % relativity and gravitation
95.55.Ym, % Gravitational radiation detectors
97.60.Lf  % black holes (astrophysics)
}

\maketitle

%%%%%%%%%%%%%%%%%%%%%%%%%%%%%%%%%%%%

\section{Introduction}
\label{sec:intro}

[Introduction]

%%%%%%%%%%%%%%%%%%%%%%%%%%%%%%%%%%%%

\section{Methodology}
\label{sec:intro}

\subsection{Fast frequency domain LISA response}
\label{sec:response}

-- Geometric setting, definitions, summarize the FD response

\subsection{Reduced order model for EOBNRv2 waveforms.}
\label{sec:waveforms}

-- Waveforms: EOBNRv2HMROM + PNext

\subsection{Fast likelihood computation}
\label{sec:likelihood}

-- Likelihood computation: fast overlaps

\subsection{Fisher matrix parameter estimation}
\label{sec:Fisher}

-- Fisher matrices computation: step self-tuning,...

\subsection{Bayesian sampling}
\label{sec:samplers}

-- Bayesian samplers: Multinest, PTMCMC

%%%%%%%%%%%%%%%%%%%%%%%%%%%%%%%%%%%%

\section{Morphology of the signals}
\label{sec:morph}

- Morphology of the response

\subsection{Temporal development of SNR}
\label{sec:timeSNR}

-- SNR accumulation with time or frequency: inspiral/MRD balance across mass range

\subsection{LISAframe parameters}
\label{sec:LISAframe}

-- LISA frame angles

\subsection{The low-frequency limit}
\label{sec:low-freq}

-- make the connection to low-f response (2 interferometers)

\subsection{A simplified likelihood model}
\label{sec:simple-like}

In this section, we investigate a simplified form of the extrinsic likelihood and the structure of its degeneracies.

It will be useful to introduce the following notation for the patter functions associated to the independent channels A and E:
\bsub\label{eq:defDaDe}
\begin{align}
	D_{a}^{+} &= \frac{1}{2} \left( 1 + \sin^{2}\betaL \right) \cos\left( 2\lambdaL - \frac{\pi}{3} \right) \,, \\
	D_{a}^{\times} &= \sin \betaL \sin\left( 2\lambdaL - \frac{\pi}{3} \right) \,, \\
	D_{e}^{+} &= - \frac{1}{2} \left( 1 + \sin^{2}\betaL \right) \sin\left( 2\lambdaL - \frac{\pi}{3} \right) \,, \\
	D_{e}^{\times} &= \sin \betaL \cos\left( 2\lambdaL - \frac{\pi}{3} \right) \,.
\end{align}
\esub
Note that the expressions for the channel E can be obtained from the expressions for A with the replacement $\lambdaL \rightarrow \lambdaL + \pi/4$.

In each channel, we separate the contribution of the modes $h_{22}$ and $h_{2,-2}$ as
\bsub\label{eq:defsase}
\begin{align}
	s_{a} &= a_{22} + a_{2,-2} \,, \\
	s_{e} &= e_{22} + e_{2,-2} \,,
\end{align}
\esub
with
\bsub\label{eq:defa22a2m2}
\begin{align}
	a_{22} &= \frac{3i}{4d} \sqrt{\frac{5}{\pi}} \cos^{4}\frac{\iota}{2} e^{2i(-\varphiL-\psiL)} \left( D_{a}^{+} + i D_{a}^{\times} \right) \,, \\
	a_{2,-2} &= \frac{3i}{4d} \sqrt{\frac{5}{\pi}} \sin^{4}\frac{\iota}{2} e^{2i(-\varphiL+\psiL)} \left( D_{a}^{+} - i D_{a}^{\times} \right) \,.
\end{align}
\esub
and similarly for the E channel with~\eqref{eq:defDe} instead of~\eqref{eq:defDa}
\bsub\label{eq:defe22e2m2}
\begin{align}
	e_{22} &= \frac{3i}{4d} \sqrt{\frac{5}{\pi}} \cos^{4}\frac{\iota}{2} e^{2i(-\varphiL-\psiL)} \left( D_{e}^{+} + i D_{e}^{\times} \right) \,, \\
	e_{2,-2} &= \frac{3i}{4d} \sqrt{\frac{5}{\pi}} \sin^{4}\frac{\iota}{2} e^{2i(-\varphiL+\psiL)} \left( D_{e}^{+} - i D_{e}^{\times} \right) \,.
\end{align}
\esub
Here, we introduced the dimensionless luminosity distance ratio $d \equiv D / D^{\rm inj}$.

In this simplified response, an exact symmetry relation is manifest: changing simultaneously
\be\label{eq:symmetryresponse}
\betaL \rightarrow -\betaL\,, \quad \iota \rightarrow \pi - \iota \,, \quad \psiL \rightarrow \pi - \psiL
\ee
leaves the signal unchanged. We can also rewrite the likelihood in a way that makes clearer the symmetry between the two pairs of parameters $(\iota, \varphiL)$ and $(\betaL, \lambdaL)$ \SM{[TODO]}.

The simplified extrinsic likelihood for a frozen LISA then takes the form
\be\label{eq:simplelikelihood}
	\ln \calL = - \frac{1}{2} \Lambda \left( \left| s_{a} - s_{a}^{\rm inj} \right|^{2} + \left| s_{e} - s_{e}^{\rm inj} \right|^{2} \right) \,,
\ee
where $\Lambda$ appears as an overall normalization factor, playing the role of the square SNR of the signal.

We can now explore possible degeneracies in the likelihood given by~\eqref{eq:simplelikelihood}, \eqref{eq:defsase} and \eqref{eq:defa22a2m2}-\eqref{eq:defe22e2m2}. We note that the relative contributions of the modes $h_{2,2}$ and $h_{2,-2}$ are strongly separated in the face-on or face-off limit, leading to a simplification of the response and an enhanced degeneracy. Due to the symmetry~\eqref{eq:symmetryresponse}, we will focus on the face-on limit $\iota \rightarrow 0$. In this limit, the $h_{2,-2}$ contributions die off rapidly as $a_{2,-2}\,, e_{2,-2} \sim \sin^{4} (\iota/2)$. If they are assumed to be negligible, then we can rewrite
\bsub\label{eq:saseapprox}
\begin{align}
	s_{a} \simeq i\calA e^{2 i \xi} \left( D_{a}^{+} + i D_{a}^{\times} \right) \,, \\
	s_{e} \simeq i\calA e^{2 i \xi} \left( D_{e}^{+} + i D_{e}^{\times} \right) \,,
\end{align}
\esub
where we introduced the notations $\xi (\varphiL, \psiL) \equiv -\varphiL - \psiL$ and $\calA(d, \iota) \equiv 3/4\sqrt{5/\pi}\cos^{4}(\iota/2)/d $ for the common phase and amplitude factors.

For a degeneracy to occur, both complex quantities $s_{a}$ and $s_{e}$ must be close to the injection values $s_{a}^{\rm inj}$ and $s_{e}^{\rm inj}$, which can be achieved as follows if~\eqref{eq:saseapprox} is valid. Defining the complex ratio
\be\label{eq:defr}
	r(\lambdaL, \betaL) = \frac{D_{a}^{+} + i D_{a}^{\times}}{D_{e}^{+} + i D_{e}^{\times} } \,,
\ee
if the position in the sky $(\lambda_{L}^{*}, \beta_{L}^{*})$ can be chosen so that
\be\label{eq:rcondition}
	r(\lambda_{L}^{*}, \beta_{L}^{*}) = \frac{s_{a}^{\rm inj}}{s_{e}^{\rm inj}} \,,
\ee
then adjusting $\calA$ and $\xi$ so that
\be
	i\calA e^{2 i \xi} = \frac{s_{a}^{\rm inj}}{\left( D_{a}^{+} + i D_{a}^{\times} \right)(\lambda_{L}^{*}, \beta_{L}^{*})}
\ee
generates degenerate points in parameter space with $\ln \calL \simeq 0$.

The conditions~\eqref{eq:defr}-\eqref{eq:rcondition} can be inverted for the sky position as follows. For a given complex value of $r$, $r(\lambda_{L}^{*}, \beta_{L}^{*}) = r$ can be rewritten as
\be\label{eq:eqtosolveforlambdabetaL}
	e^{4i\lambda_{L}^{*}}e^{i\frac{\pi}{3}} \frac{\left( 1 +\sin\beta_{L}^{*} \right)^{2}}{\left( 1 - \sin\beta_{L}^{*} \right)^{2}} = \frac{1+ i r}{1 - i r} \,.
\ee
Solving for the modulus of~\eqref{eq:eqtosolveforlambdabetaL} then yields
\be\label{eq:solutionbetaL}
	\sin\beta_{L}^{*} = \frac{\rho - 1}{\rho + 1} \,,
\ee
for
\be
	\rho = \sqrt{\left| \frac{1+ i r}{1 - i r} \right|} \,.
\ee
Solving for the argument of~\eqref{eqtosolveforlambdabetaL} gives four solutions for $\lambda_{L}^{*}$ as
\be\label{eq:solutionlambdaL}
	\lambda_{L}^{*} = - \frac{\pi}{12} + \frac{1}{4}\mathrm{Arg} \frac{1+ i r}{1 - i r} + \frac{k \pi}{2} \quad \mathrm{for} \quad k = 0,\dots,3 \,.
\ee

Repeating the argument for the other branch of solutions $\iota \rightarrow \pi$ (or directly considering the symmery~\eqref{eq:symmetryresponse}) gives four other solutions in the sky with identical sky longitudes $\lambda_{L}^{*}$, with symmetrized sky latitudes $\beta_{L}^{*} \rightarrow -\beta_{L}^{*}$, and with $\xi (\varphiL, \psiL) \equiv -\varphiL + \psiL$, $\calA(d, \iota) \equiv 3/4\sqrt{5/\pi}\sin^{4}(\iota/2)/d $.

Thus, with~\eqref{eq:solutionbetaL}, \eqref{eq:solutionlambdaL} and $\beta_{L}^{*} \rightarrow -\beta_{L}^{*}$, we arrive at eight degenerate positions in the sky $(\lambda_{L}^{*}, \beta_{L}^{*})$. For each of these sky positions, an exact line degeneracy exists between $\varphiL$ and $\psiL$ as long as $\xi = \mathrm{const}$, as well as an approximate line degeneracy between $d$ and $\iota$, as long as $\calA = \mathrm{const}$, limited to inclination values where $\sin^{4}(\iota/2)$ (or $\cos^{4}(\iota/2)$, for the other branch of solutions) remains negligible. The combination of these degeneracies generates an extended structure in the multi-dimensional parameter space. As a result, when considering the marginal posterior for the sky position, these degenerate spots in the sky can have a lot of probability support.

%%%%%%%%%%%%%%%%%%%%%%%%%%%%%%%%%%%%

\section{Illustrating examples}
\label{sec:examples}

- Demonstration/examples

\subsection{A non-degenerate case}
\label{sec:high-M-non-deg}

-- High-mass system, non-degenerate, 22 and HM + Fisher - stacked SNR contours
eg:

Case 12?
m1,    m2,    t0,D,     phi0,i,   lam,  bet, psi
8.89e5,1.11e5,0, 3.67e4,pi/3,pi/2,3pi/4,pi/3,pi/3

Case 0

-- Performance: waveform and likelihood costs, number of sampler evaluations

\subsection{Stellar origin black holes}
\label{sec:SOBH}

-- SOBH system, SNR sufficient for clean analysis + Fisher - stacked SNR contours
-- Performance: waveform and likelihood costs, number of sampler evaluations

\subsection{Highlighting degeneracies}
\label{sec:degen}

- Highlighting degeneracies

-- High-mass degenerate example, 22 vs HM - stacked SNR contours
-- show degenerate waveforms
-- simplified response and closed-form likelihood, full degeneracies
-- (explore simplified response further: parameter maps, other off-the-shelf samplers ?)

%%%%%%%%%%%%%%%%%%%%%%%%%%%%%%%%%%%%

\section{Discussion}
\label{sec:discussion}

- Summary/conclusions


----------------
Other questions:
- special orientations/sky locations ?
- Fisher matrix qualification - explore parameter choices
- continuously increase SNR - include on these examples
- continuously accumulate information over time
- full exploration of parameter space for all of the above...


%%%%%%%%%%%%%%%%%%%%%%%%%%%%%%%%%%%%

\end{document}
